% !TEX root = ../ausarbeitung.tex

\chapter{Stand der Forschung}
Hier zeigen Sie, dass Sie über Ihr Themengebiet gut informiert sind. Sie können entweder den Stand der Forschung dafür heranziehen, um Ihr Thema zu rechtfertigen (\qq{Warum ist es wichtig?}) oder Sie können die Literatur als Grundlage Ihrer Diskussion verwenden (Wie ordnen sich Ihre Beiträge in die Wissenschaftslandschaft allgemein ein?), eine Mischung ist auch möglich.

\section{Referenzen und Zitate}

Dazu gehören natürlich Referenzen zu anderen Werken. Für deren Verwaltung empfiehlt sich BibTeX oder BibLaTex, die entsprechende Datei \verb|literatur.bib| (kann natürlich auch anders benannt sein) ist in der Vorlage schon enthalten. Zur Einhaltung der Syntax bietet sich ein Online-Editor wie \cite{BibTexOnlineEditor} an; TeXstudio \cite{texstudio} hält im Menü auch Hilfe beim Erstellen der Bibliographie-Einträge bereit. Für Bücher bietet z.B. auch \url{https://books.google.de/} fertige BibTeX-Einträge an.

Die Einbindung in den Text erfolgt dann mit \verb|\cite{nielsen1994usability}| (bzw. \verb|\citep{nielsen1994usability}| für APA-Zitation mit BiBLaTex und natbib, siehe \url{https://www.overleaf.com/learn/latex/Natbib_citation_styles}), wobei der Text in den Klammern durch das in der \verb|.bib|-Datei vergebene Kürzel ersetzt werden müssen. Das Ergebnis ist dann eine Referenz zum (zumindest im Bereich \qq{Usability Engineering}) fast unverzichtbaren gleichnamigen Buch \cite{nielsen1994usability} von Jakob Nielsen.

\section{Zitationsstil}

Wie oben erwähnt, stellt diese Vorlage zwei Arten der Zitation bereit,
den numerisch-alphabeti\-schen Stil und den APA-7-Stil. Wenn Sie den
APA-Stil benutzen möchten, müssen Sie an zwei Stellen in
\verb|ausarbeitung.tex| Änderungen vornehmen:

\begin{enumerate}
\item Bei den Packages \lstinline[style=latexinline]|\usepackage[...]{biblatex}| und
  \lstinline[style=latexinline]|\addbibresource{literatur.bib}| einkommentieren, und
  \lstinline[style=latexinline]|\usepackage{babelbib}| auskommentieren.
\item Beim Literaturverzeichnis \lstinline[style=latexinline]|\printbibliography|
  einkommentieren und  \lstinline[style=latexinline]|\bibliography{literatur}| und
  \lstinline[style=latexinline]|\bibliographystyle{babplain}| auskommentieren.
\end{enumerate}

\section{Literaturverwaltung}

Für die Literaturverwaltung empfehlen wir \href{https://www.zotero.org/}{Zotero}~\cite{zotero}. In Zotero können Referenzen auch ganz einfach als BibTex oder BibLaTex exportiert werden. Hierfür empfehlen wir das Plugin \href{https://github.com/retorquere/zotero-better-bibtex}{Better BibTex for Zotero}. Am Institut für Informatik stellen wir bereits einige Literatur für Abschlussarbeiten in der Zotero-Gruppe \href{https://www.zotero.org/groups/5530859/}{
Abschlussarbeiten bei und Projekte von Heiko Holz} bereit. Sie können, nachdem Sie sich bei Zotero registriert haben, den Beitritt in die Gruppe beantragen.

Achtung: Ihre Ausarbeitung sollte sich (im Gegensatz zu diesem Template) weniger auf Internet-Quellen, als auf Bücher und Paper stützen!
\\

\hfil\rule{0.4\textwidth}{0.4pt}

\Blindtext[5]
