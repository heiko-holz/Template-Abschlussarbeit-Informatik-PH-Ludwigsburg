\documentclass[twoside,12pt,a4paper]{scrreprt} % ändern auf oneside für PDF-Abgabe

\usepackage[T1]{fontenc}
\usepackage[utf8]{inputenc}
\usepackage[ngerman]{babel}
\usepackage{babelbib}
%% Alternative, APA 7 Citation Style
%%   Dann `babelbib` oben auskommentieren
%%   und weiter unten beim Literaturverzeichnis auch noch ein-/auskommentieren
%%\usepackage[
%%  style=apa,
%%  backend=biber,
%%  sortcites=true,
%%  natbib=true,
%%  uniquename=false
%%]{biblatex}
%%\addbibresource{literatur.bib}  % deine Bib-Datei
\usepackage{parskip}
\usepackage{microtype}
\usepackage{graphicx} % Zum Einbinden von Grafiken
\usepackage[dvipsnames]{xcolor}
\usepackage[colorlinks=true,linkcolor=Black,citecolor=MidnightBlue,urlcolor=MidnightBlue]{hyperref}
\usepackage[all]{hypcap}
\usepackage{pgfplots} \pgfplotsset{compat=1.9}
\usepackage{helvet} % Schönere SansSerif-Schrift
\usepackage{times}  % Schönere Serif-Schrift

\usepackage{blindtext} % sollte am Ende nicht mehr benötigt werden ;)

\usepackage{amsmath}
\usepackage{amssymb}

\pagestyle{headings}

\graphicspath{ {figures/} } % Pfad-Prefix für einzubindende Grafiken. Es sind auch mehrere Pfade möglich, diese müssen jeweils in eigenen {Klammern} stehen.

\setkomafont{disposition}{\normalcolor\bfseries} % überall Serifen verwenden
% oder
%\renewcommand{\familydefault}{\sfdefault} % überall Sans-Serif verwenden

% PDF-Optionen (werden in den Dateieigenschaften angezeigt)
\hypersetup{
pdftitle={Titel der Arbeit},
pdfauthor={Vor- und Nachname},
pdfsubject={Bachelorarbeit Informatik},
pdfpagelayout=TwoColumnRight
}

%%% Eigene Makros
\newcommand{\qq}[1]{\glqq #1\grqq} % \qq{Text in Anführungszeichen}

\begin{document}

%%% Titelseite
\begin{titlepage}
\begin{center}
\LARGE Pädagogische Hochschule Ludwigsburg \\
\large Fakultät II: Kultur- und Naturwissenschaften \\
Institut für Informatik\\
[1cm]

\Large Bachelor-/Masterstudiengang ---~\small{(bitte Studiengang angeben)}
\\
\Large Bachelor of Arts/Master of Education/Master of Arts
\\[1cm]
\huge Bachelor-/Masterarbeit Informatik
\large\\[0.75cm]
\rule{0.5\textwidth}{1pt}
\\[0.75cm]
%[1.5cm]
\large
Genehmigungsdatum
\\[0.5cm]
\Large\textbf{Thema/Titel der Arbeit (der geht wohl in den\\ meisten Fällen über mehr als eine Zeile)}
\\\small\textrm(genauer Wortlaut siehe Genehmigung; Groß- und Kleinschreibung beachten, am Ende kein Punkt)\\
[0.5cm]
\large
Abgabedatum
\\[1cm]
\small
\textbf{Prüfer*in/nen}\\[0.3cm] 
\large Name Prüfer*in 
\\\small (mit genauem Titel und Berufsbezeichung, z.B. Jun.-Prof. Dr. Heiko Holz)
\\[.75cm]
\rule{0.5\textwidth}{1pt}
\\[0.7  5cm]
\small
\textbf{Absolvent}\\[0.3cm]
\large Name, Vorname\\
Matrikelnummer\\
E-Mail-Adresse \small(NameVorname@stud.ph-ludwigsburg.de)\\\textcolor{red}{Private E-Mail ist nicht zulässig!}
%\\[1cm]
\vfill
\fbox{%
\begin{minipage}{\textwidth}
\centering
\large
Einverständnis für Freigabe der Arbeit:\\
$\square$ ja \quad $\square$ nein (da die Arbeit personenbezogene Daten enthält)
\end{minipage}
}

\end{center}
\end{titlepage}

%%% Titelrückseite: Bibliographische Angaben
\thispagestyle{empty}
\vspace*{\fill}
\textbf{Nachname, Vorname:}\\
\emph{Titel der Arbeit}\\
Bachelor-/Masterarbeit Informatik\\
Pädagogische Hochschule Ludwigsburg\\
Bearbeitungszeitraum: Anfangs- -- Enddatum
\newpage

%%% Zusammenfassung (Abstract), hier aus externer Datei eingebunden
\input{chapters/zusammenfassung.tex}
\newpage

%%% Inhaltsverzeichnis
\KOMAoption{toc}{listof,bib} % Abbildungs-/Tabellenverzeichnis, Literaturverzeichnis aufnehmen
\tableofcontents\label{toc}
\cleardoublepage

%%% Hauptteil (mit \input{dateiname} wird die Datei 'dateiname' eingebunden)
\input{chapters/einleitung.tex}
\cleardoublepage

% !TEX root = ../ausarbeitung.tex

\chapter{Stand der Forschung}
Hier zeigen Sie, dass Sie über Ihr Themengebiet gut informiert sind. Sie können entweder den Stand der Forschung dafür heranziehen, um Ihr Thema zu rechtfertigen (\qq{Warum ist es wichtig?}) oder Sie können die Literatur als Grundlage Ihrer Diskussion verwenden (Wie ordnen sich Ihre Beiträge in die Wissenschaftslandschaft allgemein ein?), eine Mischung ist auch möglich.

Dazu gehören natürlich Referenzen zu anderen Werken. Für deren Verwaltung empfiehlt sich BibTeX oder BibLaTex, die entsprechende Datei \verb|literatur.bib| (kann natürlich auch anders benannt sein) ist in der Vorlage schon enthalten. Zur Einhaltung der Syntax bietet sich ein Online-Editor wie \cite{BibTexOnlineEditor} an; TeXstudio \cite{texstudio} hält im Menü auch Hilfe beim Erstellen der Bibliographie-Einträge bereit. Für Bücher bietet z.B. auch \url{https://books.google.de/} fertige BibTeX-Einträge an.

Die Einbindung in den Text erfolgt dann mit \verb|\cite{nielsen1994usability}| (bzw. \verb|\citep{nielsen1994usability}| für APA-Zitation mit BiBLaTex und natbib, siehe \url{https://www.overleaf.com/learn/latex/Natbib_citation_styles}), wobei der Text in den Klammern durch das in der \verb|.bib|-Datei vergebene Kürzel ersetzt werden müssen. Das Ergebnis ist dann eine Referenz zum (zumindest im Bereich \qq{Usability Engineering}) fast unverzichtbaren gleichnamigen Buch \cite{nielsen1994usability} von Jakob Nielsen.

Für die Literaturverwaltung empfehlen wir \href{https://www.zotero.org/}{Zotero}~\cite{zotero}. In Zotero können Referenzen auch ganz einfach als BibTex oder BibLaTex exportiert werden. Hierfür empfehlen wir das Plugin \href{https://github.com/retorquere/zotero-better-bibtex}{Better BibTex for Zotero}. Am Institut für Informatik stellen wir bereits einige Literatur für Abschlussarbeiten in der Zotero-Gruppe \href{https://www.zotero.org/groups/5530859/}{
Abschlussarbeiten bei und Projekte von Heiko Holz} bereit. Sie können, nachdem Sie sich bei Zotero registriert haben, den Beitritt in die Gruppe beantragen.

Achtung: Ihre Ausarbeitung sollte sich (im Gegensatz zu diesem Template) weniger auf Internet-Quellen, als auf Bücher und Paper stützen!
\\

\hfil\rule{0.4\textwidth}{0.4pt}

\Blindtext[5]

\cleardoublepage

\input{chapters/herangehensweise.tex}
\cleardoublepage

\input{chapters/evaluation.tex}
\cleardoublepage

\input{chapters/diskussion.tex}
\cleardoublepage

%%% Literaturverzeichnis, lädt die Datei literatur.bib
\bibliographystyle{babplain} % "babplain" benötigt das Paket babelbib
\bibliography{literatur}
%% Alternative, APA 7 Citation Style
%%   Dann obige beide Zeilen auskommentieren
%%\printbibliography

\cleardoublepage

%%% Selbständigkeitserklärung
\thispagestyle{empty}
\section*{Erklärung}
Hiermit erkläre ich, dass ich diese schriftliche Abschlussarbeit selbständig 
verfasst habe, keine anderen als die angegebenen Hilfsmittel und Quellen benutzt 
habe und alle wörtlich oder sinngemäß aus anderen Werken übernommenen Aussagen als 
solche gekennzeichnet habe.
\\[2cm]
Ort, Datum \hfil Unterschrift 

\end{document}